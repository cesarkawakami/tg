\documentclass[ruledheader]{abnt}
\usepackage[T1]{fontenc}
\usepackage[utf8]{inputenc}
\usepackage[portuges,brazilian]{babel}
\usepackage{graphicx}
\usepackage{amssymb, amsmath}
\usepackage{cancel}
\usepackage{enumerate}
\usepackage[aboveskip=3pt,font={small,sf},labelfont=bf]{caption}
\usepackage{subfig}
\usepackage{float}
\usepackage{fancyvrb}
\usepackage[num]{abntcite}

%%%
%%% TITLE
%%%
\autor{Cesar Ryudi Kawakami}
\titulo{Sistemas online para execução segura de código arbitrário}
\orientador{Prof. Armando Ramos Gouveia (ITA)}
\instituicao{Instituto Tecnológico de Aeronáutica}
\local{São José dos Campos}
\data{2011}

%%%
%%% PAGE SETTINGS
%%%

%%%
%%% MANY FLOATS
%%%
\renewcommand{\topfraction}{.85}
\renewcommand{\bottomfraction}{.7}
\renewcommand{\textfraction}{.15}
\renewcommand{\floatpagefraction}{.66}
\renewcommand{\dbltopfraction}{.66}
\renewcommand{\dblfloatpagefraction}{.66}
\setcounter{topnumber}{9}
\setcounter{bottomnumber}{9}
\setcounter{totalnumber}{20}
\setcounter{dbltopnumber}{9}

%%%
%%% SPECIFIC COMMANDS
%%%
\newcommand{\Lap}[1]{\mathcal{L}\left\{#1\right\}}
\newcommand{\Equiv}{\Longleftrightarrow}
\newcommand{\Impl}{\Longrightarrow}
\floatstyle{boxed}
\restylefloat{figure}
\fvset{frame=lines,fontsize=\small,xleftmargin=7pt,xrightmargin=7pt,framerule=1pt,resetmargins=true}

%%%
%%% BEGIN DOCUMENT
%%%

\begin{document}

\folhaderosto

\begin{folhadeaprovacao}
	\setlength\ABNTsignwidth{10cm}
	\setlength\ABNTsignthickness{0.4pt}
	\setlength\ABNTsignskip{3cm}
	\begin{center}
		\textbf{Sistemas online para execução segura de código arbitrário}
		
		Esta publicação foi aceita como Relatório Final de Trabalho de Graduação
	\end{center}
	\assinatura{Cesar Ryudi Kawakami \\ Autor}
	
	\assinatura{Prof. Armando Ramos Gouveia \\ Orientador}
	
	\assinatura{Prof. Carlos Henrique Costa Ribeiro \\ Coordenador do Curso de Engenharia de Computação}
	\vfill
	\begin{center}
		São José dos Campos, DATA DE APROVAÇÃO
	\end{center}
\end{folhadeaprovacao}

\begin{resumo}
Resumo em português. Lorem ipsum dolor sit amet, consectetur adipiscing elit. Fusce consequat commodo tempor. Mauris a urna vitae neque venenatis pulvinar. In sapien urna, vehicula nec ornare vel, adipiscing sed sem. Sed tristique, enim sed pulvinar eleifend, libero lectus hendrerit neque, ut dapibus ligula massa at enim. Nulla sollicitudin lectus ac magna tempus elementum. Aliquam dignissim euismod purus non accumsan. Aenean a mi urna. Nunc scelerisque rutrum hendrerit. Donec ac orci quis leo pretium tincidunt. Nunc ullamcorper diam at lectus volutpat at consequat massa sagittis. Proin nulla urna, luctus sed posuere id, mattis vitae nisl. Aliquam erat volutpat.
\end{resumo}

\begin{abstract}
Abstract in English. Lorem ipsum dolor sit amet, consectetur adipiscing elit. Fusce consequat commodo tempor. Mauris a urna vitae neque venenatis pulvinar. In sapien urna, vehicula nec ornare vel, adipiscing sed sem. Sed tristique, enim sed pulvinar eleifend, libero lectus hendrerit neque, ut dapibus ligula massa at enim. Nulla sollicitudin lectus ac magna tempus elementum. Aliquam dignissim euismod purus non accumsan. Aenean a mi urna. Nunc scelerisque rutrum hendrerit. Donec ac orci quis leo pretium tincidunt. Nunc ullamcorper diam at lectus volutpat at consequat massa sagittis. Proin nulla urna, luctus sed posuere id, mattis vitae nisl. Aliquam erat volutpat.
\end{abstract}

\tableofcontents

\listoffigures

\listoftables

\chapter{Considerações Iniciais}

Neste capítulo, são feitas considerações iniciais sobre blabla, blabla e blabla. Na seção bla, falaremos de bla.

\section{Introdução}

Um sistema online de exe blablabla possibilitaria o uso em competições de programação e em IDEs online.

\subsection{Competições de Programação}

As competições de programação, como a IOI\footnote{International Olympiad in Informatics} \cite{ioinformatics}, a ACM\footnote{Association for Computing Machinery} ICPC\footnote{International Collegiate Programming Contest} \cite{acmicpc} e a GCJ\footnote{Google Code Jam} \cite{googlecodejam}, são eventos de porte significativo com alcance mundial, posicionando-se como excelentes vetores de divulgação das ciências da computação e ensino, inspiração e captura de talentos nesse campo. Numa área do conhecimento de desenvolvimento ainda incipiente ao redor do mundo, tem-se um positivo número anual de mais de 100.000 participantes estudantis mundialmente \cite{icpcfactsheet,wang2010selection}.

O formato geral dessas competições consiste na resolução de um certo número de situações-problema (em geral, de três a doze) em um determinado intervalo de tempo, usualmente de cinco horas. A resolução de um problema, neste caso, compreende a interpretação da situação, a elaboração de uma estratégia de solução viável nos termos das restrições da situação e a correta implementação dessa solução em uma linguagem de propósito geral como C/C++, Pascal ou Java. As soluções em prova são avaliadas unicamente pela correção e eficiência de sua implementação. 

Restrições, dados e entradas dos problemas são desenhados de maneira a possibilitar a correção automatizada das soluções dos competidores, que são executadas em servidores contra uma bateria de entradas elaboradas pela banca examinadora a fim de determinar (julgar) a pontuação do competidor para aquele problema. Em competições no formato da ICPC, as soluções dos competidores são julgadas assim que recebidas de modo fornecer realimentação o mais rápido possível.

Embora cada competição empregue condições diferentes de aceitação e graduação das soluções enviadas, são dois os principais critérios utilizados. Os programas são avaliados pela sua \emph{correção} através da comparação dos resultados emitidos contra um catálogo de entradas e saídas esperadas elaborado pela comissão julgadora. Os programas são também avaliados pela sua \emph{eficiência} através do seu tempo de execução total em relação à bateria de testes executada, para a qual existe um tempo-referência máximo. Tais tempos-referência baseiam-se nas melhores soluções conhecidas para o problema em questão implementadas pela banca, e a configuração do problema é ajustada para se ter tempos máximos no entorno entre alguns segundos e alguns minutos.

Devido à elevada relação número de problemas--tempo disponível, ficando em menos de 30 minutos por problema em competições como a ICPC, são provas em que o tempo é considerado recurso escasso e de controle essencial para uma boa eficiência.

\subsection{Sistemas Informatizados para Competições de Programação}

Para a boa condução de uma competição de programação e devido a requisitos particulares inerentes à mesma, sistemas informatizados especializados tornam-se ferramentas de fundamental importância. Dentre as características necessárias a um bom sistema para competições de programação, podem ser ressaltados os seguintes pontos.

\subsubsection{Escalabilidade}

No contexto das competições de programação, devem ser considerados cenários de provas \emph{on-site}\footnote{Situações em que todos os competidores competem em um mesmo local designado.} com 500 clientes como durante a IOI \cite{ioi-nl1-2007} e provas \emph{off-site}\footnote{Situações em que a prova é praticada inteiramente através da internet, assim atingindo escala mundial mais facilmente.} com 30.000 clientes simultaneamente como durante a GCJ \cite{googlecodejamhistory}.  Considerando o padrão de uso pesado dos sistemas envolvidos, particularmente devido ao requerimento de executar por minutos pedaços de código arbitrário enviados pelos competidores, surge a necessidade de boa escalabilidade na arquitetura de execução de código do sistema em questão.

Tendo em vista, ainda, a natureza dinâmica e agitada do ambiente de prova, taxa-se também qualquer estrutura de front-end do sistema, seja pela necessidade de respostas rápidas por parte dos competidores, seja pelo próprio uso intensivo consequência da vontade dos competidores de ter a informação mais atual com grande frequência.

O desenho da arquitetura do sistema em questão deve, então, levar em conta a escalabilidade como caminho para o aumento do throughput e da velocidade de resposta considerados.

\subsubsection{Resposta Rápida e Comunicação Bidirecional}







\bibliography{tg-cesarkawakami}

\end{document}
